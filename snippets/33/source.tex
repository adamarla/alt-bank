\documentclass[14pt,fleqn]{extarticle}
\RequirePackage{prepwell-eng}

\previewoff 
\newcommand\fx{ \left(\frac{x^3}{3} + \frac{x^2}{2} \right)} 
\newcommand\gx{ \left(\frac{x^2}{2} + x \right)}

\begin{document} 
\begin{snippet}
    \incorrect
    
    If $f(x) = \fx$ and $g(x) = \gx$ then $f'(x) > g'(x)$ for $x\in (-1,1)$ 
    
    \reason
    
    \begin{align}
	f'(x) &= \ddx \fx = x^2 + x \\
	g'(x) &= \ddx \gx = x + 1 \\ 
	f'(x) &> g'(x) \implies x^2 + x > x + 1 \\
	\text{or } x^2 - 1 &> 0 \implies (x-1)\cdot (x+1) > 0 
\end{align}

Below are all the possible scenarios 
\begin{center}
  \begin{tabular}{NNNN}
   \toprule
        & x < -1 & (-1,1) & > 1 \\
   \midrule 
   x - 1 & - & - & + \\ 
    \midrule 
    x + 1 & - & + & + \\ 
    \midrule
    x^2 - 1 & + & - & + \\
    \bottomrule
  \end{tabular}
\end{center}
Therefore, if anything, $f'(x) > g'(x)$ for $x\in (-\infty, -1)\cup (1,\infty)$ 
\end{snippet} 
\end{document} 