\documentclass[14pt,fleqn]{extarticle}
\RequirePackage{prepwell}
\previewoff
\begin{document}

\newcommand\sq{ \left(v + \frac{1}{2} \right)^2 + \left( \frac{\sqrt{3}}{2}\right)^2} 
\newcommand\dnm{ \left(v^2 + v + 1\right)}

%text
Solve the following differential equation 
\[ \quad \left(x-y \right)\frac{dy}{dx} = x + 2y \]%

\newcard

$\left(x-y \right)\dfrac{dy}{dx} = x + 2y$ is a homogenous differential equation 

\newcard 

$\left(x-y \right)\dfrac{dy}{dx} = x + 2y$ is a first-order linear differential equation 

\newcard 

Before we can solve a differential equation, we must first know what kind of differential equation it is 

\begin{align}
	\left(x-y \right)\frac{dy}{dx} & = x + 2y \implies \frac{dy}{dx} = \frac{x+2y}{x-y} \\ 
	&\text{or } \frac{dy}{dx} = \frac{1 + 2\left(\frac{y}{x} \right)}{1- \frac{y}{x}} = F \left(\frac{y}{x} \right) 
\end{align}

$\frac{dy}{dx} = F \left(\frac{y}{x} \right)$ is what a \underline{homogenous} differential equation looks like. Hence, the given differential equation is homogenous 

\newcard 

If $ y = v\cdot x$, then 
\[ \qquad \frac{dx}{x} = \frac{1-v}{v^2+v+1}\cdot dv \]

\newcard 

If $ y = v\cdot x$, then 
\[ \qquad \frac{dx}{x} = \frac{1+v}{v^2+v+1}\cdot dv \]

\newcard 

For homogenous diffrential equations, we set $y = v\cdot x$. And hence, 

\begin{align}
\frac{dy}{dx} =  \frac{1+2 \left(\frac{y}{x} \right)}{1 - \frac{y}{x}} &\implies \frac{d}{dx} \left(v\cdot x \right) = \frac{1+2v}{1-v} \\
\text{or } \underbrace{v + x \frac{dv}{dx}}_{\text{Product Rule}} &= \frac{1+2v}{1-v} \\
x \frac{dv}{dx} &= \frac{1+2v}{1-v} - v = \frac{v^2+v+1}{1-v} \\
\therefore \frac{dx}{x} &= \frac{1-v}{v^2+v+1}\cdot dv
\end{align}

\newcard 

\begin{align}
\text{If } z &= \dnm \text{ then } \\
\int \frac{dx}{x} &= \int \frac{1-v}{\dnm}\cdot dv \\ 
&= \int \frac{A\cdot \frac{d}{dx} \dnm + B }{\dnm}\cdot dv  \\
&= \underbrace{-\frac{1}{2}\int \frac{dz}{z} + \frac{3}{2}\int \frac{dv}{\dnm}}_{A = -\frac{1}{2}\text{ and } B = \frac{3}{2}}
\end{align}

\newcard 

Notice that the degree of the numerator $(1-v)$ is just \underline{one less} than the degree of the denominator $\dnm$. At such times, it is best to 
\small\[ \qquad \text{Numerator} = A \frac{d}{dx}\text{(Denominator)} + B \]\normalsize

Which is why 
\begin{align}
\int \frac{dx}{x} &= \int \frac{1-v}{v^2+v+1}\cdot dx \\ 
\text{If } 1 - v &= A\cdot \frac{d}{dv}\dnm + B \\
\text{then } 1- v &= A\cdot (2v+1) + B \\
\implies A &= -\frac{1}{2}\text{ and } B = \frac{3}{2}\\
\therefore I &= \int \frac{1-v}{\dnm}\cdot dv \\
&= -\frac{1}{2}\int \frac{dz}{z} + \frac{3}{2}\int \frac{dv}{\dnm} \\
\text{where } z &= \dnm 
\end{align}

\newcard 

\begin{align}
\text{In }I &= \underbrace{-\frac{1}{2}\int \frac{dz}{z}}_P + \underbrace{\frac{3}{2}\int \frac{dv}{\dnm}}_Q + C \\
P &= \log \sqrt{\frac{x^2}{y^2 + xy + x^2}} 
\end{align}

\newcard 

\begin{align}
	I &= \underbrace{-\frac{1}{2}\int \frac{dz}{z}}_P + \underbrace{\frac{3}{2}\int \frac{dv}{\dnm}}_Q + C \\
	P &= -\frac{1}{2}\log \vert z\vert = -\frac{1}{2}\log \vert\dnm\vert \\
	&= -\frac{1}{2}\log\left\vert \left(\frac{y}{x} \right)^2 + \frac{y}{x} + 1 \right\vert \\
	&= -\frac{1}{2}\log\left\vert \frac{y^2 + xy + x^2}{x^2}\right\vert  \\
	&= \underbrace{\frac{1}{2}\log\left\vert \frac{x^2}{y^2+xy+x^2}\right\vert}_{\log \frac{1}{b} = -\log b } \\
	&= \underbrace{\log \sqrt{\frac{x^2}{y^2+xy+x^2}}}_{ a\log b = \log b^a }
\end{align}

\newcard 

\begin{align}
\text{And in }I &= \underbrace{-\frac{1}{2}\int \frac{dz}{z}}_P + \underbrace{\frac{3}{2}\int \frac{dv}{\dnm}}_Q + C \\
Q &= \sqrt{3}\tan^{-1} \left(\frac{2y+x}{\sqrt{3} x } \right)
\end{align}

Which gives 
\small\[ I = \log \sqrt{\frac{x^2}{y^2+xy+x^2}} + \sqrt{3}\tan^{-1} \left(\frac{2y+x}{\sqrt{3} x } \right) + C\]\normalsize

\newcard 

\begin{align}
Q &= \underbrace{\frac{3}{2}\int \frac{dv}{\dnm} = \frac{3}{2}\int \frac{dv}{\sq}}_{\text{Complete the squares}} \\
&= \frac{3}{2}\cdot \frac{1}{\frac{\sqrt{3}}{2}}\cdot\tan^{-1} \left(\frac{v+\frac{1}{2}}{\frac{\sqrt{3}}{2}} \right) \\
&= \sqrt{3}\cdot\tan^{-1} \left(\frac{2v+1}{\sqrt{3}} \right) = 
\sqrt{3}\tan^{-1} \left(\frac{2y+x}{\sqrt{3} x } \right)
\end{align}

And therefore 
\small\[ I = \log \sqrt{\frac{x^2}{y^2+xy+x^2}} + \sqrt{3}\tan^{-1} \left(\frac{2y+x}{\sqrt{3} x } \right) + C \]\normalsize

\end{document}