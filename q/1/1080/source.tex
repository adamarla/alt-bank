\documentclass[14pt,fleqn]{extarticle}
\RequirePackage{prepwell-eng}
\previewoff

\newcommand\nextrd{\frac{4}{9}}
\newcommand\bwins{\frac{2}{9}}
\newcommand\expa{ \left(\nextrd \right) }

\begin{document}
\begin{problem}
\statement
	

      $A$ and $B$ throw a die alternatively till one
      of them gets a number greater than $4$ and
      wins the game. If $A$ starts the game, then
      what is the probability of $B$ winning?

\begin{step}
  \begin{options} 
     \correct 
     
     Let a round be defined as a single roll of the dice by $A$ followed by a single roll of dice by $B$. Given this \newline 
     
     \begin{center}
  \begin{tabular}{cN}
   \toprule
       Event & \text{Probability} \\
   \midrule 
   No one wins in a round & \nextrd \\
    \midrule 
    $B$ wins in the $N-$th round & \left(\nextrd \right)^{N-1}\cdot\bwins \\
    \bottomrule
  \end{tabular}
\end{center}
       
     \incorrect
     
     Let a round be defined as a single roll of the dice by $A$ followed by a single roll of dice by $B$. Given this \newline 
     
     \begin{center}
  \begin{tabular}{cN}
   \toprule
       Event & \text{Probability} \\
   \midrule 
   No one wins in a round & \frac{2}{9} \\
    \midrule 
    $B$ wins in the $N-$th round & \left(\frac{2}{9} \right)^{N-1}\cdot\frac{2}{3} \\
    \bottomrule
  \end{tabular}
\end{center}
        
    \end{options} 
     \reason 
       
     For $B$ to win, $A$ should \underline{not} already have won the game by getting a number $> 4$\newline 
     
     Hence, for $B$ to win \underline{in the $N-$th round}, no one should
     have won in the \underline{preceding} $N-1$ rounds\newline 
    
    \begin{center}
  \begin{tabular}{cN}
   \toprule
        Event & \text{Probability} \\
   \midrule 
   Single dice roll $> 4$ & \frac{2}{6} = \frac{1}{3} \\
    \midrule 
    Single dice roll $\leq 4$ & 1-\frac{1}{3} = \frac{2}{3} \\
    \midrule 
    No one wins in a round & \frac{2}{3}\cdot\frac{2}{3} = \nextrd\\
    \midrule
    $B$ wins in a round & \underbrace{\frac{2}{3}\cdot\frac{1}{3} = \bwins}_{\text{$A$ doesn't win but $B$ does}} \\ 
    \midrule
    $B$ wins in the $N-$th round & \left(\nextrd \right)^{N-1}\cdot\frac{2}{9} \\
    \bottomrule
  \end{tabular}
\end{center} 
    
\end{step}    

\begin{step}
  \begin{options} 
     \correct 
       
      The probability that $B$ wins is therefore 
      \[ \qquad P = \sum_{N=1}^\infty \left(\nextrd \right)^{N-1}\cdot\bwins = \frac{2}{5} \]
        
    \end{options} 
     \reason 
     
     $B$ could win in the $1-$st round or the $100-$th round or the $1000-$th round. 
     Hence, the \underline{total probability} of $B$ winning is 
     
     \begin{align}
     P &= \expa^0\bwins + \expa^1\bwins + \expa^2\bwins + \cdots \infty \\
     &= \sum_{N=1}^\infty\expa^{N-1}\cdot\bwins \\
     &= \bwins \left[\dfrac{1}{1-\nextrd} \right] = \bwins\cdot\frac{9}{5} = \frac{2}{5}
\end{align}

Notice that $B$ only has a $40\%$ chance of winning compared to $A$'s $60\%$ simply because $A$ rolls first 
       
\end{step}

\end{problem}
\end{document}
