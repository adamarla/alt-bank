\documentclass[14pt,fleqn]{extarticle}
\RequirePackage{prepwell-eng}

\newcommand\expa{\left(x^3-x \right)}
\newcommand\expb{ \left[\frac{x^4}{4} - \frac{x^2}{2} \right]}
\newcommand\expc{ \left[\frac{x^2}{2} - \frac{x^4}{4} \right]}

\newcommand\fx{\vert\, x^3 - x\,\vert}
\newcommand\intg{\int_{-1}^2 } 

\previewoff 

\begin{document} 
\begin{problem}
	\statement 
    
     Evaluate the following 
     \[ \qquad \intg\fx\cdot dx  \]  
     
     \begin{step}
  \begin{options} 
     \correct 
       
     From $-\infty\to\infty, f(x) = \expa$ behaves as follows 
     \begin{center}
  \begin{tabular}{NNNNN}
   \toprule
        &  x<-1 & x\in \left[-1,0 \right) & x\in \left[0,1 \right] & x > 1 \\
   \midrule 
   f(x) & < 0 & \geq 0 & \leq 0 & > 0 \\
    \bottomrule
  \end{tabular}
\end{center}
       
     \incorrect
     
      From $-\infty\to\infty, f(x) = \expa$ behaves as follows 
     \begin{center}
  \begin{tabular}{NNNNN}
   \toprule
        &  x<-1 & x\in \left[-1,0 \right) & x\in \left[0,1 \right] & x > 1 \\
   \midrule 
   f(x) & < 0 & \leq 0 & \geq 0 & > 0 \\
    \bottomrule
  \end{tabular}
\end{center}
        
    \end{options} 
     \reason 
     
     \begin{align}
     f(x) &= \expa = x\cdot (x^2-1) \\
     &= x\cdot (x-1)\cdot (x+1) 
\end{align}
 
By looking at how each \underline{individual term} is in \underline{different intervals}, we can know how $f(x)$ behaves in those intervals 
\begin{center}
  \begin{tabular}{NNNNN}
   \toprule
        &  x & x-1 & x+1 & f(x) = \expa \\
   \midrule 
   x < -1 & - & - & - & - \\
    \midrule 
    x\in \left[-1,0\right) & - & - & + & + \\
    \midrule 
    x\in \left[0,1 \right] & + & - & + & - \\
    \midrule 
    x > 1 & + & + & + & + \\
    \bottomrule
  \end{tabular}
\end{center}
       
\end{step}

\begin{step}
  \begin{options} 
     \correct 
     
     And therefore 
     \begin{align}
	A &= \intg \fx\cdot dx \\
	&= \int_{-1}^0 \expa\cdot dx + \int_0^1 \left(x-x^3 \right)\cdot dx \\
	&+ \int_1^2 \expa\cdot dx 
\end{align}
        
    \end{options} 
     \reason 
     
     We established in the last step that 
     \begin{center}
  \begin{tabular}{NNNNN}
   \toprule
        &  x < -1 & x\in \left[-1,0 \right) & x\in \left[0,1 \right] & x > 1\\
   \midrule 
   \expa & - & + & - & + \\
    \midrule 
    \fx & x-x^3 & x^3 - x & x-x^3 & x^3 - x \\
    \bottomrule
  \end{tabular}
\end{center}
Which is why 
\begin{align}
	A &= \intg \fx\cdot dx \\
	&= \int_{-1}^0 \expa\cdot dx + \int_0^1 \left(x-x^3 \right)\cdot dx \\
	&+ \int_1^2 \expa\cdot dx 
\end{align}
       
\end{step}

\begin{step}
  \begin{options} 
     \correct 
     
     Which gives $A = \dfrac{7}{2} $   
       
     \incorrect
     
     Which gives $A = \dfrac{5}{2}$
        
    \end{options} 
     \reason 
     
     \smallmath
\begin{align}
	A &= \int_{-1}^0 \expa\cdot dx + \int_0^1 \left(x-x^3 \right)\cdot dx \\
	&+ \int_1^2 \expa\cdot dx \\
	&= \expb_{-1}^0  + \expc_0^1 + \expb_1^2 \\
	&= \left[0-\left(-\frac{1}{2} \right) \right] + \left[\left(\frac{1}{2} \right)-0 \right] + \left[2- \left(-\frac{1}{2} \right) \right] \\
	&= \frac{7}{2}
\end{align}
       
\end{step}
\end{problem} 
\end{document} 
