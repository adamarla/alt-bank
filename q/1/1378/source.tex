\documentclass[14pt,fleqn]{extarticle}
\RequirePackage{prepwell}
\previewoff
\begin{document}

%text
For what value of k is the following 
function continuous at $x=2$? 

%
\[\qquad f(x) = \begin{cases} 
2x + 1\, ;  x < 2 \\
k \, ;  x = 2 \\
3x - 1\, ;  x > 2
\end{cases}\]

\newcard

%text
For $f(x)$ to be continuous at $x=2$
\[ \quad \lim_{x\to 2^-}f(x) = \lim_{x\to 2^+}f(x) = k \]
%

\newcard

%text
For $f(x)$ to be continuous at $x=2$
\[ \quad \lim_{x\to 2^-}f(x) = \lim_{x\to 2^+}f(x)  \]
%

\newcard

%text
\[ \lim_{x\to 2^-}f(x) = \lim_{x\to 2^+}f(x) \] only ensures
that $f(x)$ has a limit at $x = 2$\newline 

 But for $f(x)$ to be continuous, there must be \underline{no gap at $x=2$}\newline

 And therefore, 
 \[ \quad \lim_{x\to 2^-}f(x) = \lim_{x\to 2^+}f(x) = k \]

%

\newcard

\begin{align}
\lim_{x\to 2^-} f(x) &= \lim_{x\to 2^-}(2x + 1) = 5 \\
\lim_{x\to 2^+} f(x) &= \lim_{x\to 2^+}(3x - 1) = 5 
\end{align}
%text

Therefore, $k = 5$ for $f(x)$ to be 
continuous at $x = 2$
%

\newcard

\begin{align}
\lim_{x\to 2^-} f(x) &= \lim_{x\to 2^-}(3x-1) = 5 \\
\lim_{x\to 2^+} f(x) &= \lim_{x\to 2^+}(2x+1) = 5 
\end{align}
%text

Therefore, $k = 5$ for $f(x)$ to be 
continuous at $x = 2$
%

\newcard

%text
The answer $k = 5$ is correct\newline

But if you got this step wrong, then you didn't \underline{read $f(x)$ properly}\newline

Notice that for $x < 2, f(x) = 2x + 1$.
And for $x > 2, f(x) = 3x - 1$\newline

Hence, 
%
\begin{align}
\lim_{x\to 2^-} f(x) &= \lim_{x\to 2^-}(2x + 1) = 5 \\
\lim_{x\to 2^+} f(x) &= \lim_{x\to 2^+}(3x - 1) = 5
\end{align}

\end{document}