\documentclass[14pt,fleqn]{extarticle}
\RequirePackage{prepwell-eng}
\previewoff

\newcommand\fxa{\sqrt{\frac{1+x}{2}}}


\begin{document}

\begin{problem}
	\statement 
    
     Find the derivative of the following function with respect to $x$ 
     at $ x = 1$ 
     \[ \qquad y = \cos^{-1} \left(\sin\fxa \right) + x^x \] 
     
     \begin{step}
  \begin{options} 
      
     \correct 
     
      To evaluate $\ddx A$ in the equation below       
	\[ \qquad y = \underbrace{\cos^{-1} \left(\sin\fxa \right)}_A + \underbrace{x^x}_B \]
	
	\underline{Strategy $II$ is better} (see below) 
	
	\begin{center}
  \begin{tabular}{ll}
   \toprule
        \text{Strategy $I$} & \text{Strategy $II$}  \\
   \midrule 
   Let $x = \cos 2\theta$ & Use $\cos^{-1} z + \sin^{-1} z = \frac\pi{2}$ \\
    \bottomrule
  \end{tabular}
\end{center}

     \incorrect

     To evaluate $\ddx A$ in the equation below       
	\[ \qquad y = \underbrace{\cos^{-1} \left(\sin\fxa \right)}_A + \underbrace{x^x}_B \]
	
	\underline{Strategy $I$ is better} (see below) 
	
	\begin{center}
  \begin{tabular}{ll}
   \toprule
        \text{Strategy} $I$ & \text{Strategy $II$}  \\
   \midrule 
   Let $x = \cos 2\theta$ & Use $\cos^{-1} z + \sin^{-1} z = \frac\pi{2}$ \\
    \bottomrule
  \end{tabular}
\end{center}
	
	

      
    \end{options} 
     \reason 
       
     You can try Strategy $I$. But it wouldn't make life any easier 
     \begin{align}
	   A &= \cos^{-1} \left(\sin \fxa \right) \\
	   &= \cos^{-1} \left(\sin \sqrt{\frac{1+\cos 2\theta}{2}} \right) \\
	   &= \cos^{-1} \left(\sin \cos\theta \right)
\end{align}

This doesn't look very helpful\newline 

So, let us try Strategy $II$ 

\begin{align}
A &= \cos^{-1} \underbrace{\left(\sin\fxa \right)}_z \\
&= \frac\pi{2} - \sin^{-1} \left(\sin\fxa \right) \\
&= \frac\pi{2} - \fxa 
\end{align}

This is a whole lot simpler 
\end{step}

\begin{step}
  \begin{options} 
     \correct 
     
     \begin{center}
  \begin{tabular}{NN}
   \toprule
        \ddx A & \ddx B \\
   \midrule 
   \frac{-1}{2\sqrt{2\cdot (1+x)}} & x^x\cdot \left(\log x + 1 \right)\\
    \bottomrule
  \end{tabular}
\end{center}  

Therefore $\ddx y = \frac{3}{4}$ when $x = 1$ 

     \incorrect
     
     \begin{center}
  \begin{tabular}{NN}
   \toprule
        \ddx A & \ddx B \\
   \midrule 
   \frac{1}{2\sqrt{2\cdot (1+x)}} & x\cdot x^{x-1} = x^x \\
    \bottomrule
  \end{tabular}
\end{center}  

Therefore $\ddx y = \frac{5}{4}$ when $ x = 1$ 

    \end{options} 
     \reason 
      
     \begin{align}
     y = A + B &\implies \ddx y = \ddx A + \ddx B \\
	A &= \frac{\pi}{2} - \fxa \\
	\therefore \ddx A &= 0 - \frac{1}{\sqrt{2}}\cdot \frac{1}{2}\cdot \frac{1}{\sqrt{1+x}} \\
	\text{And }B &= x^x \implies \log B = x \log x \\
	\therefore \ddx \log B &= \ddx \left(x\cdot\log x \right) \\
	\text{or } \frac{1}{B}\ddx B &= \underbrace{\frac{x}{x} + \log x}_{\text{Product Rule}} \\
	\therefore \ddx B &= x^x\cdot \left(1 + \log x \right)
\end{align}  

Hence $\ddx y$ at $x = 1$ equals 
\begin{align}
\left[\ddx y \right]_{x=1} &= \left[\ddx A + \ddx B \right]_{x=1} \\ 
&= \left[- \frac{1}{2\sqrt{2\cdot 2}} + 1^1\cdot \left(1 + \log 1 \right)  \right] \\
&= 1 - \frac{1}{4} = \frac{3}{4}
\end{align}
\end{step}
\end{problem} 

\end{document}
