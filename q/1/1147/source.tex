\documentclass[14pt,fleqn]{extarticle}
\RequirePackage{prepwell}
\previewoff
\begin{document}

A card from a pack of 52 playing cards is lost. From the remaining 
cards, three cards are drawn at random without replacement and are found 
to be all spades. Find the probability of the lost card being a spade 


\newcard

\begin{center}
  \begin{tabular}{NNN}
   \toprule
        \text{Event} & \text{Meaning} & \text{Probability}  \\
   \midrule 
   L & \text{Card lost is a spade} & \frac{1}{4} \\
    \bottomrule
  \end{tabular}
\end{center}

\newcard 

There are four suites with 13 cards each in a full deck of 52 cards\newline 

Hence, if a card is lost, then it must belong to one of the four suites\newline 

Which is why $P(L) = \frac{1}{4}$ 

\newcard 

Given the following event 
\begin{center}
  \begin{tabular}{Nl}
   \toprule
       \text{Event} &  Meaning\\
   \midrule
   S_3 & All three cards drawn are spades \\ 
    \bottomrule
  \end{tabular}
\end{center}

we know the following 

\begin{center}
  \begin{tabular}{NNN}
   \toprule
         \condp{S_3}{L}  & \quad & \condp{S_3}{L'} \\
   \midrule 
          \frac{12}{51}\cdot \frac{12}{51}\cdot \frac{12}{51} & \quad & 
          \frac{13}{51}\cdot \frac{13}{51}\cdot \frac{13}{51} \\
    \bottomrule
  \end{tabular}
\end{center}

\newcard 

Given the following event 
\begin{center}
  \begin{tabular}{Nl}
   \toprule
       \text{Event} &  Meaning\\
   \midrule
   S_3 & All three cards drawn are spades \\ 
    \bottomrule
  \end{tabular}
\end{center}

we know the following 

\begin{center}
  \begin{tabular}{NNN}
   \toprule
         \condp{S_3}{L}  & \quad & \condp{S_3}{L'} \\
   \midrule 
          \frac{12}{51}\cdot \frac{11}{50}\cdot \frac{10}{49} & \quad & 
          \frac{13}{51}\cdot \frac{12}{50}\cdot \frac{11}{49} \\
    \bottomrule
  \end{tabular}
\end{center}


\newcard 

Once a card is lost, we either have 12 spades or 13 spades to start with. This gives 
rise to two sets of probabilities -- $\condp{S_3}{L}$ and $\condp{S_3}{L'}$ \newline 

Moreover, the cards are drawn without replacement. Which means 
that with each card drawn, there is one less card the next time \newline 

And that is why the probability of drawing \underline{three successive spades} is as follows 

\begin{center}
  \begin{tabular}{NNN}
   \toprule
         \condp{S_3}{L}  & \quad & \condp{S_3}{L'} \\
   \midrule 
          \frac{12}{51}\cdot \frac{11}{50}\cdot \frac{10}{49} & \quad & 
          \frac{13}{51}\cdot \frac{12}{50}\cdot \frac{11}{49} \\
    \bottomrule
  \end{tabular}
\end{center}

\newcard 

The required probability is $\condp{L}{S_3}$ 

\newcard 

The required probability is $P \left(L\cap S_3 \right)$ 

\newcard

$S_3$ has already happened. And we want to know the probability of $L$ having happened\newline 

And therefore, the required probability is 
\[ \qquad \qquad \condp{L}{S_3} \]

\newcard 

\begin{align}
	\condp{L}{S_3} &= \fcondp{L}{S_3} = \frac{10}{49} 
\end{align}

\newcard 

As per Bayes' Theorem 
\begin{align}
	\condp{L}{S_3} &= \fcondp{L}{S_3}
\end{align} 

where 
\[\underbrace{P(S_3) = \condp{S_3}{L}P(L)+ \condp{S_3}{L'}P(L')}_{\text{Theorem of Total Probability}} \]

And therefore, 
\begin{align}
	\condp{L}{S_3} &= \dfrac{\left(\frac{12}{51}\cdot\frac{11}{50}\cdot\frac{10}{49}\right)\cdot\frac{1}{4}}
            {\left(\frac{12}{51}\cdot\frac{11}{50}\cdot\frac{10}{49}\right)\cdot\frac{1}{4} + 
            \left(\frac{13}{51}\cdot\frac{12}{50}\cdot\frac{11}{49}\right)\cdot\frac{3}{4}} \\
            &= \dfrac{10}{49}
\end{align}
	
\end{document}