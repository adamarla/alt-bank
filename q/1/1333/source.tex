\documentclass[14pt,fleqn]{extarticle}
\RequirePackage{prepwell-eng}
\previewoff

\newcommand\third{\frac{1}{3}} 

\begin{document}

\begin{problem}
	\statement 
    
    Three bags contain balls as shown in the table below 
    
    \begin{center}
  \begin{tabular}{NNNN}
   \toprule
        \text{Bag} & \text{White} & \text{Black} & \text{Red} \\
   \midrule 
   B_1 & 1 & 2 & 3 \\
    \midrule 
    B_2 & 2 & 1 & 1 \\
    \midrule 
    B_3 & 4 & 3 & 2 \\
    \bottomrule
  \end{tabular}
\end{center}
A bag is chosen at random and two balls are drawn from it.
They happen to be red and white. What is the probability 
that the balls came from $B_3$? 

\begin{step}
  \begin{options} 
     \correct 
       
       If we define the following events 

\begin{center}
  \begin{tabular}{Nc}
   \toprule
   \text{Event} & \text{Meaning}\\
\midrule
	A & \text{$1$ red} \& \text{$1$ white ball drawn}\\
\midrule
	B_1 & \text{Bag $B_1$ picked}\\
	\midrule
	B_2 & \text{Bag $B_2$ picked}\\
	\midrule
	B_3 & \text{Bag $B_3$ picked}\\
\bottomrule
\end{tabular}
\end{center}
then we need to find $\condp{B_3}{A}$ 

     \incorrect
        
          If we define the following events 

\begin{center}
  \begin{tabular}{Nc}
   \toprule
   \text{Event} & \text{Meaning}\\
\midrule
	A & \text{$1$ red} \& \text{$1$ white ball drawn}\\
\midrule
	B_1 & \text{Bag $B_1$ picked}\\
	\midrule
	B_2 & \text{Bag $B_2$ picked}\\
	\midrule
	B_3 & \text{Bag $B_3$ picked}\\
\bottomrule
\end{tabular}
\end{center}
then we need to find $\prob{B_3\cap A}$ 
    \end{options} 
     \reason 
     
     \underline{If neither $A$ nor $B_3$} have happened, then yes, it would 
make sense to evaluate $\prob{B_3\cap A}$ \newline  

But $A$ has already happened. Hence, the probability we are looking 
for is $\condp{B_3}{A}$ 
       
\end{step}

\begin{step}
  \begin{options} 
     \correct 
     
     
     \begin{align}
	\condp{B_3}{A} &= \fcondp{B_3}{A} \\
	&= \dfrac{\condp{A}{B_3}\cdot \prob{B_3}}{\sum_{k=1}^3 \condp{A}{B_k}\cdot \prob{B_k}}
\end{align}

where $\prob{B_k}$ be the probability of choosing the $k-$th bag
       
    \end{options} 
     \reason 
      
      This is simply the mathematical formulation of Bayes' Theorem 
      \begin{align}
      \condp{B_3}{A} &= \fcondp{B_3}{A} \\
      \text{where } \prob{A} &= \underbrace{\sum_{k=1}^3\condp{A}{B_k}\cdot \prob{B_k}}_{\text{Theorem of Total Probability}}
\end{align} 

Notice that in order to evaluate $\condp{B_3}{A}$, we will first need to 
find $\condp{A}{B_k}$ for $k=1,2,3$ 
\end{step}

\begin{step}
  \begin{options} 
     \correct 
       
       Given the number of balls of each colour in each bag, we can infer the 
       following 
       \begin{center}
  \begin{tabular}{NN}
   \toprule
       \prob{B_1} = \third & \condp{A}{B_1} = \frac{1}{5} \\
   \midrule 
   \prob{B_2} = \third & \condp{A}{B_2} =  \frac{1}{3} \\
    \midrule 
    \prob{B_3} = \third & \condp{A}{B_3} =  \frac{2}{9} \\
    \bottomrule
  \end{tabular}
\end{center}
       
     \incorrect
        
         Given the number of balls of each colour in each bag, we can infer the 
       following 
       \begin{center}
  \begin{tabular}{NN}
   \toprule
       \prob{B_1} = \third & \condp{A}{B_1} = \frac{1}{10} \\
   \midrule 
   \prob{B_2} = \third & \condp{A}{B_2} =  \frac{1}{6} \\
    \midrule 
    \prob{B_3} = \third & \condp{A}{B_3} =  \frac{1}{9} \\
    \bottomrule
  \end{tabular}
\end{center}
        
    \end{options} 
     \reason 
     
     All three bags are \underline{equally likely} to be picked. Hence 
     $P \left(B_1 \right) = P \left(B_2 \right) = P \left(B_3 \right) = \frac{1}{3}$ \newline 

The (slightly) tricky bit is finding $\condp{A}{B_k}$ -- the probability of 
drawing a red and a white ball from the $k-$th bag \newline 

Now, if $B_k$ has $R$ red balls, $W$ white balls and $T$ total balls, then 
\[ \qquad \condp{A}{B_k} = \dfrac{^RC_1\times ^WC_1}{^TC_2} \] 

And hence
\begin{center}
  \begin{tabular}{NNN}
   \toprule
	\text{Bag} & \text{Total Balls} & \condp{A}{B_k} \\
\midrule
	B_1 & 6 & \dfrac{^3C_1\times ^1C_1}{^6C_2} = \frac{1}{5} \\
\midrule
	B_2 & 4 & \dfrac{^2C_1\times ^1C_1}{^4C_2} = \frac{1}{3} \\
\midrule
	B_3 & 9 & \dfrac{^4C_1\times ^2C_1}{^9C_2} = \frac{2}{9} \\
\bottomrule
  \end{tabular}
\end{center}

       
\end{step}

\begin{step}
  \begin{options} 
     \correct 
       
       And therefore 
       \[ \condp{B_3}{A} = \frac{5}{17} \]
     \incorrect
     
            And therefore 
       \[ \condp{B_3}{A} = \frac{17}{45} \]
        
    \end{options} 
     \reason 
       
     We know everything we need to find $\condp{B_3}{A}$. It is all 
     calculations from here onwards
     
     \begin{align}
	\condp{B_3}{A} &= \dfrac{\condp{A}{B_3}\cdot \prob{B_3}}{\sum_{k=1}^3 \condp{A}{B_k}\cdot \prob{B_k}} \\
	&= \underbrace{\left[\dfrac{\frac{1}{3}\cdot \frac{2}{9}}{\frac{1}{3}\cdot \left(\frac{1}{5}+\frac{1}{3}+\frac{2}{9} \right)} \right]}_{\prob{B_1}=\prob{B_2} = \prob{B_3} = \frac{1}{3}} = \frac{5}{17}
\end{align}  
\end{step}

\end{problem} 
\end{document}
