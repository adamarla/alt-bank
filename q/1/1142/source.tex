\documentclass[14pt,fleqn]{extarticle}
\RequirePackage{prepwell}
\previewoff
\begin{document}
\newcard

From a lot of $15$ bulbs of which $5$ are defective, a sample of $4$ 
bulbs in drawn one by one with replacement\newline 

Find the probability distribution of number of defective bulbs. 
Hence, find the mean of the distribution 

\newcard

\begin{center}
  \begin{tabular}{NlN}
   \toprule
        \text{Probability} &  Of & \text{Value} \\
   \midrule 
   p & Drawing a single  & \frac{1}{3} \\
   & defective bulb & \\
    \midrule 
    q & Drawing a single  & \frac{2}{3} \\
    &non-defective bulb & \\
    \bottomrule
  \end{tabular}
\end{center}

\newcard 

\begin{center}
  \begin{tabular}{NlN}
   \toprule
        \text{Probability} &  Of & \text{Value} \\
   \midrule 
   p & Drawing a single  & \frac{1}{2} \\
   & defective bulb & \\
    \midrule 
    q & Drawing a single  & \frac{1}{2} \\
    &non-defective bulb & \\
    \bottomrule
  \end{tabular}
\end{center}

\newcard 

$5$ out $15$ bulbs are defective. Hence, the probability $p$ 
of drawing a single defective bulb is $p= \frac{5}{15} = \frac{1}{3}$ \newline 

And therefore, the probability of drawing a non-defective bulb is $q= 1-p = \frac{2}{3}$ 


\newcard 

The probability of drawing \underline{exactly one} defective bulb is 
\[ \qquad P_1 = ^4C_1\cdot p\cdot q^3 \]

Similarly, the probability of drawing \underline{exactly} $N$ defective bulbs is 
\[ \qquad P_N = ^4C_N\cdot p^N \cdot q^{4-N}\quad (N \leq 4) \]

\newcard 

The probability of drawing \underline{exactly one} defective bulb is 
\[ \qquad P_1 = p\cdot q^3 \]

Similarly, the probability of drawing \underline{exactly} $N$ defective bulbs is 
\[ \qquad P_N = p^N \cdot q^{4-N}\quad (N \leq 4) \]

\newcard 

Below are all the ways in which one could draw \underline{exactly one} defective 
bulb $(X$) in four draws $(I-IV)$ 

\begin{center}
  \begin{tabular}{NNNN}
   \toprule
        I & II & III & IV \\
   \midrule 
   X & \checkmark & \checkmark & \checkmark  \\
   \midrule
    \checkmark & X & \checkmark & \checkmark  \\
    \midrule 
    \checkmark & \checkmark & X & \checkmark  \\
    \midrule
    \checkmark & \checkmark & \checkmark & X  \\
    \bottomrule
  \end{tabular}
\end{center} 

Notice that there are $^4C_1$ ways of drawing one defective bulb \newline 

Also, remember that the bulbs are drawn with replacement. Which means the bulb is put back after being drawn -- and hence the probability of drawing a defective bulb in the next draw is the same as before \newline 

And hence the total probability of drawing exactly one defective bulb is 
\[ \qquad P_1 = ^4C_1\cdot p\cdot q^3 \]

Similarly, the probability of drawing \underline{exactly} $N$ defective bulbs is 
\[ \qquad P_N = ^4C_N\cdot p^N \cdot q^{4-N}\quad (N \leq 4) \]

\newcard 

The probability distribution of drawing $N$ defective bulbs $(0\leq N\leq 4)$ is therefore 

 \begin{center}
\begin{tabular}{NNNNN}
      \toprule
      P_0 & P_1 & P_2 & P_3 & P_4 \\
      \midrule
      \frac{16}{81} & \frac{32}{81} & \frac{8}{27} & \frac{8}{81} & \frac{1}{81} \\
      \bottomrule
      \end{tabular}
\end{center}

\newcard 

The probability distribution table is simply a table that lists the probabilities 
of all possible outcomes -- in this case the getting $0-4$ defective bulbs 

\begin{center} 
\begin{tabular}{NN}
        \midrule 
        N &  P(n=N) \\
        \midrule 
        0 & ^4C_0\cdot p^0q^4 =\frac{16}{81} \\
        \midrule
        1 &  ^4C_1\cdot p^1q^3=\frac{32}{81} \\
        \midrule
        2 & ^4C_2\cdot p^2q^2 =\frac{8}{27} \\
        \midrule
        3 &  ^4C_3\cdot p^3q = \frac{8}{81} \\
        \midrule
        4 & ^4C_4\cdot p^4q^0 =\frac{1}{81} \\
        \midrule      
      \end{tabular}
\end{center}


\newcard

\[ \text{Expected value} = \sum_{k=0}^4 P_k\cdot k = \frac{4}{3} \]

\newcard

We know the number of defective bulbs we can draw $(0-4)$ and the probabilities of drawing them 

\begin{center}
\begin{tabular}{cNNNNN}
      \toprule
      & k=0 & 1 & 2 & 3 & 4 \\
      \midrule
      Probability $(P_k)$ & \frac{16}{81} & \frac{32}{81} & \frac{8}{27} & \frac{8}{81} & \frac{1}{81} \\
      \midrule 
      Value $(k)$ & 0 & 1 & 2 & 3 & 4 \\
      \bottomrule
      \end{tabular}
\end{center}

Hence, the expected value $E(x)$ is simply 

\smallmath
\begin{align}
E(x) &= 0\cdot \frac{16}{81} + 1\cdot \frac{32}{81} + 2\cdot \frac{8}{27} + 3\cdot\frac{8}{81} + 4\cdot \frac{1}{81}\\
&= \frac{4}{3}	
\end{align}
\end{document}